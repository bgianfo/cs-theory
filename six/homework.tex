\documentclass[11pt]{article}

\oddsidemargin0cm
\topmargin1cm     %I recommend adding these three lines to increase the 
\textwidth16.5cm   %amount of usable space on the page (and save trees)
\textheight23.5cm  

\newcommand{\question}[2] {\vspace{.25in} \fbox{#1} #2 \vspace{.10in}}
\renewcommand{\part}[1] {\vspace{.10in} {\bf (#1)}}
\usepackage{pstricks,pst-node,pst-tree}
\usepackage{graphicx}
\usepackage{amssymb,amsmath}
\usepackage{multicol}
\usepackage{subfig}

\begin{document}

\medskip                        % Skip a "medium" amount of space
                                % (latex determines what medium is)
                                % Also try: \bigskip, \littleskip

\begin{center}                  % Center the following lines
  {\Large Introduction to the Theory of Computation \\ Homework \#6} \\
  Brian Gianforcaro \\
  \date \\
\end{center}

%\ttfamily

\question{1}{2.2}

\part{a} Example 2.36 uses the pumping lemma for Context Free Languages that the 
language \indent $Z = \{ a^n b^n c^n | n \geq 0\}$ is not context free. The language 
$ A = \{ a^m b^n c^n | m,n\geq 0\}$ given can be \indent generated by the following grammar.

$ S \to XY $

$ X \to aY \; | \; \epsilon$

$ Y \to bYc \; | \; \epsilon$


Thus $ A $ is obviously context-free. We can similarly generate a grammar for the language given  
\indent $ B = \{ a^n b^n c^m | m,n\geq 0\}$  which is also context free.

Finally $A \cap B = Z$ which is not context-free.

\part{b}

$ C = \overline{\overline C} = \overline{\overline{A \cap B}} == \overline{\overline A \cap \overline B}  $

However, then C will be context free.

If $ \overline A$ and $ \overline B $ are context free then so are, $\overline A \cap \overline B $.

Thus $\overline{\overline A \cap \overline B}$  is context free by these assumptions.

Therefore the class of CFG's are NOT closed under complementation.

\question{2}{2.9}
 
$ S \to WY \;|\; ZX \;$

$ W \to aW \;|\; \epsilon $

$ X \to cX \;|\; \epsilon $

$ Y \to bYc \;|\;\epsilon $

$ Z \to aZb \;|\;\epsilon $

Yes the CFG is ambiguous, two different possible derivation's exist for a string like "abc".


\question{3}{Sipser: 2.5 }

\part{b} 

\psmatrix 
[mnode=circle,colsep=.85cm,rowsep=1cm] 
% States: 
[mnode=R]{\mbox{}} 
& & & & & [name=1]$q_1$ \\ 
& & & & [name=2]$q_2$ & & [name=3]$q_3$ \\[0pt] 
& & & & [doubleline=true,name=4]$q_4$ &  &[doubleline=true,name=5]$q_5$ & & 
\endpsmatrix 
% Transitions: 
\psset{nodesep=3pt,arrows=->,arcangle=15, labelsep=2pt,shortput=nab} 
\footnotesize 
sncline[linestyle=solid]{1,1}{1} 
\ncline{1}{2}_{$ \epsilon,\epsilon \to 1 $} 
\ncline{1}{3}^{$ \epsilon,\epsilon \to 0 $} 
\ncline{2}{4}_{$ \epsilon,\epsilon \to 1 $} 
\ncline{3}{5}_{$ \epsilon,\epsilon \to 0 $} 
\nccircle[angleA=270]{3}{.4cm}_{  $ \epsilon,\epsilon \to  0$ ; $ \epsilon,\epsilon \to 1 $}
\nccircle[angleA=-270]{2}{.4cm}_{ $ \epsilon,\epsilon \to  0$ ; $ \epsilon,\epsilon \to 1 $}

\pagebreak 

\question{4}{Give informal descriptions and PDA.}

\part{a} The same number of 0's and 1's no matter what the order.  \\ \\
\psmatrix 
[mnode=circle,colsep=.85cm,rowsep=1cm] 
% States: 
[mnode=R]{\mbox{}} 
& & & & [name=22]$q_2$ \\[0pt] 
& [doubleline=true,name=11]$q_1$ \\ 
& & & & [name=33]$q_3$ & & 
\endpsmatrix 
% Transitions: 
\psset{nodesep=3pt,arrows=->,arcangle=15, labelsep=2pt,shortput=nab} 
\footnotesize 
\ncline[linestyle=solid]{1,1}{11} 
\ncarc{11}{22}^{$ 0,\epsilon \to \$ $} 
\ncarc{22}{11}^{$ 1,\$ \to \epsilon  $} 

\ncarc{11}{33}^{$ 1,\epsilon \to \$ $} 
\ncarc{33}{11}^{$ 0,\$ \to \epsilon  $} 

\nccircle[angleA=270]{33}{.4cm}_{  $ 0, 1 \to \epsilon$ ; $ 1,\epsilon \to 1 $}
\nccircle[angleA=270]{22}{.4cm}_{ $ 0,\epsilon \to  0$ ; $ 1, 0 \to \epsilon $}


\part{b} Any combination of 0's and 1's as long as their are not the same number of each. \\ \\

\psmatrix 
[mnode=circle,colsep=.85cm,rowsep=1cm] 
% States: 
[mnode=R]{\mbox{}} 
& & & & & & & [name=222]$q_2$  \\[0pt] 
& [doubleline=true,name=111]$q_1$ \\ 
& & & & & & & [name=333]$q_3$ & & 
\endpsmatrix 
% Transitions: 
\psset{nodesep=1pt,arrows=->,arcangle=10, labelsep=2pt,shortput=nab} 
\footnotesize 
\ncline[linestyle=solid]{1,1}{111} 
\ncarc{111}{222}^{$ 0,\epsilon \to \$ $} 
\ncarc{222}{111}^{$ 1,\$ \to \epsilon  $} 
\ncarc{222}{333}^{$ 1,\epsilon \to 0 $} 
\ncarc{333}{222}^{$ 0,\epsilon \to 1 $} 
\ncarc{111}{333}^{$ 1,\epsilon \to \$ $} 
\ncarc{333}{111}^{$ 0,\$ \to \epsilon  $} 
\nccircle[angleA=270]{333}{.4cm}_{  $ 0, 1 \to \epsilon$ ; $ 1,\epsilon \to 1 $}
\nccircle[angleA=270]{222}{.4cm}_{ $ 0,\epsilon \to  0$ ; $ 1, 0 \to \epsilon $}

\end{document}
