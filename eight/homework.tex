\documentclass[11pt]{article}

\oddsidemargin0cm
\topmargin0cm     %I recommend adding these three lines to increase the 
\textwidth16.5cm   %amount of usable space on the page (and save trees)
\textheight23.5cm  

\newcommand{\question}[2] {\vspace{.25in} \fbox{#1} #2 \vspace{.10in}}
\renewcommand{\part}[1] {\vspace{.10in} {\bf (#1)}}
\usepackage{pstricks,pst-node,pst-tree}
\usepackage{pstricks-add}
\usepackage{graphicx}
\usepackage{amssymb,amsmath}

\defineTColor{TRed}{red}
\defineTColor{TGreen}{green}
\defineTColor{TBlue}{blue}
\defineTColor{TYellow}{yellow}
\defineTColor{TOrange}{orange}
\defineTColor{TGray}{gray}


\begin{document}

\medskip                        % Skip a "medium" amount of space
                                % (latex determines what medium is)
                                % Also try: \bigskip, \littleskip

\begin{center}                  % Center the following lines
  {\Large Introduction to the Theory of Computation \\ Homework \#8} \\
  Brian Gianforcaro \\
  \date \\
\end{center}

%\ttfamily

\question{1}{Venn Diagram:}

\begin{frame}
\leavevmode\put(0,0){%
  \begin{pspicture}(-3,-3)(2,5)
    \pscircle[linecolor=green,style=TGreen](1;45){4cm}
    \pscircle[linecolor=gray,style=TGray](8;05){4cm}
    \pscircle[linecolor=red,style=TRed](0.8;165){1cm}
    \pscircle[linecolor=blue,style=TBlue](-2.2;185){1.7cm}
    \pscircle[linecolor=orange,style=TOrange](0.8;163){1.7cm}
    \rput(8,1.5){\textbf{Recursive Languages}}
    \rput(-0.65,1.5){\textbf{CFLs}}
    \rput(-0.65,0.5){\textbf{NFA}}
    \rput(2.3,0.5){\textbf{$\overline{R.E.}$}}
    \rput(0.6,3.3){\textbf{Turing Machine, 25 tapes}}
  \end{pspicture}
}
\end{frame}

\question{2}{\textbf{Proof:}}

\smallskip

\begin{itemize}

\item M = "On input $x$ check if $x = (G)$ where $G$ is an CFG .
  \begin{enumerate}  
  \centering
  \raggedright
    \item Convert $G$ into Chomsky normal form.
    \item Mark all non-terminals $NT$ which have some rule $NT \to n$
    \item Repeat until no new non-terminals are marked:
      \begin{enumerate}
        \item Mark the non-terminal $NT$ if there is a rule
        \item $NT \to XZ$ such that $Z$ and $X$ are already marked.
      \end{enumerate}
    \item If $W$ is marked $(L(G) \not = \emptyset)$ then $M$ rejects, else accepts."
  \end{enumerate}
\item Since decider M can be constructed for the problem: "Given a CFG $G$, is $L(G)$ not empty?" then it must be decidable
\end{itemize}


\indent $\Box$

\pagebreak
\question{3}{\textbf{Proof:}}

\begin{itemize}
\item Assume $B$ is countable and that $f\colon \mathbb{N} \to  \mathbb{B}$ is a functional correspondence. 
\item Let $f_{i}(j)$ denote the $i^{th}$ symbol in $f(j)$. 
\item Consider the string $s$ whose $i^{th}$ element differs from $f_{i}(i)$, then the  $i^{th}$ symbol in $s$ is $1$ and vice versa.
\item Then $s$ cannot be in the image of $f$ since it differs from every string in the image of $f$ by at least one symbol. 
\item Furthermore, $s$ is in $B$ since it is an infinite sequence over $\{0,1\}$.
\item Therefore, B is uncountable.
\end{itemize}

\indent $\Box$

\question{4}{\textbf{Proof:}}

\begin{itemize}
  \item $M$ = "On input $<R,S>$ where $R$ and $S$ are regular expressions 
 \begin{center}
   \begin{enumerate}
     \item Convert $R$ to DFA $A$ and $S$ to DFA $B$ 
     \item Construct DFA $C$ such that $L(C) = L(B) ? L(A)$ 
     \item Submit $<A,C>$ to the decider for EQDFA 
     \item If it accepts, $accept$.
     \item If it rejects, $reject$." 
   \end{enumerate}
 \end{center}
     
  \item $M$ is a decidable since steps $1$, $2$, $4$, and $5$ will not create and infinite loops and step $2$ calls a decider. 
  \item Also, $M$ accepts $<R,S>$ if and only if $L(R) = L(R)$ ? $L(S).$ 
  \item Therefore, $M$ is a decider for $A$ so $A$ is decidable. 
\end{itemize}
\indent $\Box$

\question{5}{\textbf{Problem:} Given a CFG G, is $L(G) = \Sigma^{*}$}

\part{a} CFGs are strictly more powerful than REs, this problem is in both $RE$ and $\overline{RE}$.

\part{b} \textbf{Proof Sketch:} 
\begin{itemize}
  \item Given that $ALL_{CFG}$ is is undecidable, $L(M_1) = L(M_2)$ is a similar problem.
  \item As it asks if each FA produces the same set of all possible strings. 
  \item As the set of strings is countably infinite, the question is answerable but it would in essence never return.
  \item For example any case similar to $a^*$. 
  \item In summary, the problem is atleast as hard as $ALL_{CFG}$, and would create a endless loop. 
  \item Thus the problem is undecidable.
\end{itemize}
\indent \indent $\Box$

\pagebreak
\question{6}{\textbf{Proof:}}

\begin{itemize}
\item We show that the language $T$ is undecidable by showing that $H_{TM} \geq T$. 
\item The reduction function $f$ takes as input $(M,w)$ where $M$ is a Turing Machine and $w$ is a string. 
\item $f$ outputs $f(M,w) = (N_{M,w})$, where $N_{M,w}$ is the Turing Machine that does the following:

 \begin{center}
  \begin{itemize}
    \item On input $x$:
    \begin{enumerate}
     \item If $x = 01$, then $accept$.
     \item Else if $x \not = 10$, then $reject$.
     \item Else then simulate $M$ on $w$. If $M$ halts on $w$, then $accept$.
    \end{enumerate}
  \end{itemize}
 \end{center}

\item Clearly, the function $f$ is computable. If $(M,w) \in H_{TM}$ or $M$ halts on $w$ then $L(N_{M,w}) = \{01,10\}$, thus $(N_{M,w}) \notin T$. 
\item We conclude that $(M,w) \in H_{TM}$ if and only if $f(M,w) \in T$. 
\item Therefore $f$ is a reduction from $H_{TM}$ to $T$. It follows that $T$ is undecidable as $H_{TM}$ is undecidable.
\end{itemize}

\indent $\Box$

\question{7}{\textbf{Proof:}}

\begin{itemize}
\item We show that $USELESS_{TM}$ is undecidable by showing that $\overline{H_{TM}} \geq USELESS_{TM}$.
\item The reduction function $f$ takes as input $(M,w)$ where $M$ is a Turing Machine and $w$ is a string. 
\item $f$ outputs $f(M,w) = (N_{M,w})$, where $N_{M,w}$ is the Turing Machine that does the following:
 \begin{center}
  \begin{itemize}
    \item On input symbol $0$, $N_{M,w}$ simply enters all it's states except for a special state $\overline{q}$.
    \item On input symbol $1$, $N_{M,w}$ simulates $M$ on $w$, if $M$ halts on $w$, then enters state $\overline{q}$.
  \end{itemize}
 \end{center}

\item Clearly, the function $f$ is computable. 
\item From the behavior of  $N_{M,w}$ on input symbol $0$, it is clear that only the special state $\overline{q}$ can be useless. 
\item From the behavior of $N_{M,w}$  on input symbol 1, it is clear that $\overline{q}$ is useless if and only if $M$ does not halt on $w$.
\item Therefore $(M,w) \in \overline{H_{TM}}$ if and only if $f(M,w) \in USELESS_{TM}$, and hence f is a reduction from $\overline{H_{TM}}$ to $USELESS_{TM}$. 
\item It follows that $USELESS_{TM}$ is undecidable as $H_{TM}$ is undecidable.
\end{itemize}

\indent $\Box$

\end{document}
