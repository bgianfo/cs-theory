\documentclass[11pt]{article}

\oddsidemargin0cm
\topmargin1cm     %I recommend adding these three lines to increase the 
\textwidth16.5cm   %amount of usable space on the page (and save trees)
\textheight23.5cm  

\newcommand{\question}[2] {\vspace{.25in} \fbox{#1} #2 \vspace{.10in}}
\renewcommand{\part}[1] {\vspace{.10in} {\bf (#1)}}
\usepackage{pstricks,pst-node,pst-tree}
\usepackage{graphicx}
\usepackage{amssymb,amsmath}
\usepackage{multicol}
\usepackage{subfig}

\begin{document}

\medskip                        % Skip a "medium" amount of space
                                % (latex determines what medium is)
                                % Also try: \bigskip, \littleskip

\begin{center}                  % Center the following lines
  {\Large Introduction to the Theory of Computation \\ Homework \#7} \\
  Brian Gianforcaro \\
  \date \\
\end{center}

%\ttfamily

\question{1}{Proof: By Mathematical Induction}
 \begin{center}
  \begin{itemize}
   \item Let $T$ be any binary tree.
   \item Let $height(x)$ be the height of any binary tree x.
   \item Let $leaves(x)$ be the number of leaves in any binary tree x.

   \item Suppose for any $n \in N$ if $height(T) < n$ then, $leaves(T) < 2^n$

   \item \textbf{Basis Step:}

   \item We have the binary tree $T$ with one node, and no leaves.
   
   \item Since $height(T) = 0$ and $leaves(T) = 0$ then $0 < n$ is true and $ 0 < 2^n$ is also true. 

   \item \textbf{Induction:}

   \item Assume an $n \in N$.

   \item Let $T$ be any arbitrary tree of size $s$, and by definition of a binary tree T is the root with two sub-tree's.

   \item Let $height(T)$ of a binary tree be defined as: $height(T) = 1 + max( height(T_L), height(T_R) ) $

   \item Since $leaves(T)$ must be an integer it can be said that: $leaves(T) \leq \lceil \frac{s}{2} \rceil $

   \item Accordingly it can also be said that $height(T) \leq \lceil \frac{s}{2} \rceil$

   \item Thus if  $\lceil \frac{s}{2} \rceil < n$ then, $ \lceil \frac{s}{2} \rceil < 2^n$ must also hold.
   
   \item Therefore we have established for any $n \in N$ if $height(T) < n$ then, $leaves(T) < 2^n$
  \end{itemize}
 \end{center}

\indent $\Box$

\pagebreak

\question{2}{Proof: By Contradiction}

 \begin{center}
  \begin{itemize}

    \item Suppose $L = \{ a^{i} | i $ is a square $ \}$ is a CFL.

    \item Then the pumping lemma applies and there exists a number n so that any sufficiently long string $z \in L$ can be pumped.

    \item Pick $z = a^n$

    \item Since $|z| \geq n$, it follows that $z = uvwzy$, $|vwx| \leq n$, and $|vx| > 0 $.

    \item Since $|vwx| \leq n$, $vx$ can contain at most the two different symbols. Further, if there are two symbols they must be adjacent.

    \item Suppose $xv$ contains one symbol. Then that symbol can be increased and so $a^n$ is in $L$ for $k > n$. But none of those are in $L$.

    \item Thus we arrive at a contradiction.

    \item So Suppose $vx$ contains two symbols. Then two adjacent symbols can be increased by the same amount.
    
    \item If such an increase leads to alteration, we immediately get a contradiction.
    
    \item Otherwise, only adjacent symbols are increased, and nonadjacent symbols are left balanced again leading to a contradiction. 

    \item For example $a a^n \notin L$ or $ a^{11} a^n \notin L$.

    \item Thus in all cases, we obtain a contradiction so $L$ is a CFL can not be the case.

  \end{itemize}
\end{center}
\indent\indent $\Box$

\question{3}{Are the following languages CFLs?}

\part{a} Yes

\part{b} Yes

\part{c} No

\part{d} Yes

\part{e} No

\pagebreak

\question{4}{Sipser, excercise 3.1}

\begin{multicols}{4}

\part{c} 000.

$ $

$ \vdash_m q_1 0 0 0 \sqcup $

$ \vdash_m \sqcup q_2 0 0 \sqcup $

$ \vdash_m \sqcup x q_3 0 \sqcup $

$ \vdash_m \sqcup x 0 q_4 \sqcup $

$ \vdash_m q_{reject} $

\part{d} 000000.


$ $ 

$ \vdash_m q_1 0 0 0 0 0 0  $

$ \vdash_m \sqcup q_2 0 0 0 0 0 $

$ \vdash_m \sqcup x q_3 0 0 0 0 $

$ \vdash_m \sqcup x 0 q_4 0 0 0 $

$ \vdash_m \sqcup x 0 x q_3 0 0 $

$ \vdash_m \sqcup x 0 x 0 q_4 0 $

$ \vdash_m \sqcup x 0 x 0 x q_3 $

$ \vdash_m \sqcup x 0 x 0 x q_3 $

$ \vdash_m \sqcup x 0 x 0 q_5 x $

$ \vdash_m \sqcup x 0 x q_5 0 x $

$ \vdash_m \sqcup x 0 q_5 x 0 x $

$ \vdash_m \sqcup x q_5 0 x 0 x $

$ \vdash_m q_5 \sqcup x 0 x 0 x $

$ \vdash_m \sqcup q_2 x 0 x 0 x $

$ \vdash_m \sqcup x q_2 0 x 0 x $

$ \vdash_m \sqcup x x q_3 x 0 x $

$ \vdash_m \sqcup x x x q_3 0 x $

$ \vdash_m \sqcup x x x 0 x q_4 $

$ \vdash_m q_{reject} $

\end{multicols}

\question{5}{Sipser, exercies 3.2}

\begin{multicols}{2}


\part{c} 1\#\#1

$ $ 

$ \vdash_m q_1 1 \# \# 1 $

$ \vdash_m x q_3 \# \# 1 $

$ \vdash_m x \# q_5 \# 1 $

$ \vdash_m q_{reject} $

\part{d} 10\#\#11

$ $ 

$ \vdash_m q_1 1 0 \# \# 1 1 $

$ \vdash_m x q_3 0 \# \# 1 1 $

$ \vdash_m x 0 q_3 \# \# 1 1 $

$ \vdash_m x 0 \# q_5 \# 1 1 $

$ \vdash_m q_{reject} $

\end{multicols}
\question{6}{A machine which accepts a's then b's. The number of b's have to be greater than the number of a's but less than twice the number of a's}

$ $

$ $

$ $

$ $

\psmatrix 
[mnode=circle,colsep=1.2cm,rowsep=1cm] 
% States: 
[mnode=R]{\mbox{}} 
& & & & & [name=0]$q_0$ \\ \\
& & & & & [name=1]$q_1$  & & & & &[name=4]$q_4$ \\[0pt] 
& &[name=2]$q_2$ \\
& [doubleline=true,name=a]$q_a$ & & & [name=3]$q_3$ & &
\endpsmatrix 
% Transitions: 
\psset{nodesep=3pt,arrows=->,arcangle=15, labelsep=4pt,shortput=nab} 
\footnotesize 
\ncline[linestyle=solid]{0,0}{1} 
\ncline{4}{0}_{$ \#, \# \to R $} 
\ncline{0}{2}_{$ b \to R $} 
\ncline{0}{1}^{$ a, \# \to R $} 
\ncline{1}{2}_{$ b \to R $} 
\ncline{1}{3}_{$ \# \to L $} 
\ncline{2}{a}_{$ \# \to R $} 
\ncarc{3}{4}_{$ b, \# \to L $} 
\nccircle[angleA=290]{1}{.25cm}_{\textbf{ $ b \to R $}}
\nccircle[angleA=-290]{1}{.25cm}_{ $ a \to R$ }
\nccircle[angleA=200]{2}{.4cm}_{ $ b \to R$ }
\nccircle[angleA=-200]{4}{.4cm}_{ $ a \to L$ }
\nccircle[angleA=270]{4}{.4cm}_{ $ b \to L$ }

\end{document}
