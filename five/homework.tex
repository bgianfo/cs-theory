\documentclass[11pt]{article}

\oddsidemargin0cm
\topmargin-2cm     %I recommend adding these three lines to increase the 
\textwidth16.5cm   %amount of usable space on the page (and save trees)
\textheight23.5cm  

\newcommand{\question}[2] {\vspace{.25in} \fbox{#1} #2 \vspace{.10in}}
\renewcommand{\part}[1] {\vspace{.10in} {\bf (#1)}}

\usepackage{pst-node}
\usepackage{graphicx}
\usepackage{amssymb,amsmath}

\begin{document}

\medskip                        % Skip a "medium" amount of space
                                % (latex determines what medium is)
                                % Also try: \bigskip, \littleskip


\begin{center}                  % Center the following lines
  {\Large Introduction to the Theory of Computation \\ Homework \#5} \\
  Brian Gianforcaro \\
  \date \\
\end{center}

\ttfamily


\question{1}{Proof:}
\begin{center}
  \begin{itemize}
    \item Let $L = \{ a^{i} b^{j} | i \ne j \} $ is regular.

    \item L is regular can not be the case.
  \end{itemize}
\end{center}

\indent $\Box$




\question{2}{
Let $x$ and $y$ be strings and let $L$ be any language. We say that $x$ and $y$ are distin- 
guishable by $L$ if some string z exists whereby exactly one of the strings $xz$ and $yz$ 
is a member of $L$; otherwise, for every string $z$, we have $xz \in L$ whenever $yz \in L$ 
and we say that $x$ and $y$ are indistinguishable by $L$. If $x$ and $y$ are indistinguishable 
by $L$ we write $x {\equiv}_{L} y$. Show that ${\equiv}_{L}$ is an equivalence relation.
  }

\question{3}{}

\question{4}{Minimize the DFA.}
 
\begin{center}
\begin{tabular}{|c|c|c|c|c|c|c|c|c|c|}
\hline
$\;$&$1$&$2$&$3$&$4$&$5$&$6$&$7$&$8$&$9$\\
\hline
1&-&-&-&-&-&-&-&-&-\\
\hline
2&$x_1$&-&-&-&-&-&-&-&-\\
\hline
3&$x_1$&$x_1$&-&-&-&-&-&-&-\\
\hline
4&$x_1$&\;&$x_1$&-&-&-&-&-&-\\
\hline
5&$x_0$&$x_0$&$x_0$&$x_0$&-&-&-&-&-\\
\hline
6&$x_0$&$x_0$&$x_0$&$x_0$&\;&-&-&-&-\\
\hline
7&$x_1$&$x_1$&$x_1$&$x_1$&$x_0$&$x_0$&-&-&-\\
\hline
8&$x_1$&$x_1$&$x_1$&$x_1$&$x_0$&$x_0$&\;&-&-\\
\hline
9&$x_0$&$x_0$&$x_0$&$x_0$&\;&\;&$x_0$&$x_0$&-\\
\hline
\end{tabular}
\end{center}

\begin{figure}[!htb]
\centering
\psframebox*[fillcolor=black!10!cyan!20,fillstyle=solid]{%
\rule[-1.7cm]{0pt}{3.9cm}
$\psmatrix[mnode=Circle,radius=5mm,colsep=2.0cm,rowsep=1.5cm,
	arrowscale=1.5]
[name=1,style=Cblue] 1 &  [name=2,shadow=true] 2 &  [name=3,style=Cred] 3
\ncline[nodesep=1pt]{->}{1}{2}_{S_o\lambda^{-1}}
\ncline[nodesep=1pt]{->}{2}{3}_{S_a\lambda^{-1}}
\ncarc[nodesep=1pt,arcangle=60]{<-}{1}{3}^{F_a\lambda^{-1}}
\nccurve[angleA=-105,angleB=-55,ncurv=4,nodesep=1pt]{->}{3}{3}
\nbput[nrot=:U]{S_a\lambda^{-1}}
\endpsmatrix$ }

\caption{This is an example of a life-cycle diagram drawn using PStricks}
\end{figure}


\question{5}{ }

\part{a}
 
\part{b}

\part{c}

\part{d}

\question{6}{Sipser 2.4(b): $\{ w | w $ starts and ends with the same symbol$\}$ }

$ R \rightarrow 0 R \;|\; 1 R\; |\; \epsilon $


$ S \rightarrow 0\; |\; 1 \;|\; 0 R 0 \;|\; 1 R 1$


\end{document}
