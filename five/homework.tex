\documentclass[11pt]{article}

\oddsidemargin0cm
\topmargin1cm     %I recommend adding these three lines to increase the 
\textwidth16.5cm   %amount of usable space on the page (and save trees)
\textheight23.5cm  

\newcommand{\question}[2] {\vspace{.25in} \fbox{#1} #2 \vspace{.10in}}
\renewcommand{\part}[1] {\vspace{.10in} {\bf (#1)}}
\usepackage{pstricks,pst-node,pst-tree}
\usepackage{graphicx}
\usepackage{amssymb,amsmath}
\usepackage{multicol}
\usepackage{subfig}

\begin{document}

\medskip                        % Skip a "medium" amount of space
                                % (latex determines what medium is)
                                % Also try: \bigskip, \littleskip

\begin{center}                  % Center the following lines
  {\Large Introduction to the Theory of Computation \\ Homework \#5} \\
  Brian Gianforcaro \\
  \date \\
\end{center}

%\ttfamily

\question{1}{Proof: By contradiction}
\begin{center}
  \begin{itemize}
    \item Assume $L = \{ a^{i} b^{j} | i \ne j \} $ is regular language.
    \item Let p be the pumping length.
    \item Now consider the string $s = a^{p}1^{p}$ which is in L.
    \item Using the pumpig lemma, $s = xyz$ so thate $|xy| = p$
    \item This means that $y$ consists entirely of $a$'s.
    \item Considering $xz$, we have equal the number $a$'s and $b$'s.
    \item This means the resulting string is still in the language $L$.
    \item Hence, L is regular can not be the case.
  \end{itemize}
\end{center}

\indent $\Box$


\question{2}{
Let $x$ and $y$ be strings and let $L$ be any language. We say that $x$ and $y$ are distin- 
guishable by $L$ if some string z exists whereby exactly one of the strings $xz$ and $yz$ 
is a member of $L$; otherwise, for every string $z$, we have $xz \in L$ whenever $yz \in L$ 
and we say that $x$ and $y$ are indistinguishable by $L$. If $x$ and $y$ are indistinguishable 
by $L$ we write $x {\equiv}_{L} y$. Show that ${\equiv}_{L}$ is an equivalence relation.
  }

\begin{itemize}
\item Let $L$ be a language.
\item Let $z$ be a set of strings.
\item $z$ is pairwise distinguishable by $L$ if any two strings in $z$ are distinguishable by $L$.
\item The index of $L$ is equal to the number of equivalence classes in $L$, which can be either finite or infinite.
\end{itemize}


\question{3}{ We can define a recursive set which infinetly generates all possible strings. }

\begin{itemize}
\item $\epsilon \in S$
\item $ S = \{\; ()x \;$ or $\; x() \;$ or $(x)\; | \; x \in S \;\}$
\end{itemize}

\pagebreak

\question{4}{Minimize the DFA.}

\begin{center}
\begin{tabular}{|c|c|c|c|c|c|c|c|c|c|}
\hline
$\;$&$1$&$2$&$3$&$4$&$5$&$6$&$7$&$8$&$9$\\
\hline
1&-&-&-&-&-&-&-&-&-\\
\hline
2&$x_1$&-&-&-&-&-&-&-&-\\
\hline
3&$x_1$&$x_1$&-&-&-&-&-&-&-\\
\hline
4&$x_1$&$x_1$&$x_1$&-&-&-&-&-&-\\
\hline
5&$x_0$&$x_0$&$x_0$&$x_0$&-&-&-&-&-\\
\hline
6&$x_0$&$x_0$&$x_0$&$x_0$&\;&-&-&-&-\\
\hline
7&$x_1$&$x_1$&$x_1$&$x_1$&$x_0$&$x_0$&-&-&-\\
\hline
8&$x_1$&$x_1$&$x_1$&$x_1$&$x_0$&$x_0$&\;&-&-\\
\hline
9&$x_0$&$x_0$&$x_0$&$x_0$&\;&\;&$x_0$&$x_0$&-\\
\hline
\end{tabular}
\end{center}

\begin{figure}[!htb]
\centering
\psframebox*[fillcolor=white,fillstyle=solid]{%
\rule[-1.7cm]{0pt}{3.9cm}
$\psmatrix[mnode=circle,radius=5mm,colsep=1.0cm,rowsep=1.0cm, arrowscale=1.5]
           [name=1]\{ 1 \} & [name=2] \{2\} &&  [name=3] \{ 3 \} & [name=4] \{4\} & [name=78] \{7,8\} & [name=569,shadow=true] \{5,6,9\}
\ncline[nodesep=1pt]{->}{3}{4}_{b}
\nccurve[angleA=45,angleB=125,ncurv=1,nodesep=1pt]{->}{3}{569}
\nbput[nrot=:U]{a}
\nccurve[angleA=45,angleB=135,ncurv=1,nodesep=1pt]{->}{4}{569}
\nbput[nrot=:U]{b}
\nccurve[angleA=35,angleB=155,ncurv=1,nodesep=1pt]{->}{1}{3}
\nbput[nrot=:U]{b}
\nccurve[angleA=-25,angleB=210,ncurv=1,nodesep=1pt]{->}{2}{4}
\nbput[nrot=:U]{a}
\nccurve[angleA=-35,angleB=220,ncurv=1,nodesep=1pt]{->}{2}{569}
\nbput[nrot=:U]{b}
\nccurve[angleA=-35,angleB=-155,ncurv=1,nodesep=1pt]{->}{1}{2}
\nbput[nrot=:U]{a}
\nccurve[angleA=210,angleB=-35,ncurv=1,nodesep=2pt]{->}{569}{78}
\nbput[nrot=:D]{b}
\nccurve[angleA=-105,angleB=-55,ncurv=3,nodesep=1pt]{->}{569}{569}
\nbput[nrot=:U]{a}
\nccurve[angleA=35,angleB=155,ncurv=1,nodesep=1pt]{->}{78}{569}
\nbput[nrot=:U]{b}
\nccurve[angleA=-105,angleB=-55,ncurv=3,nodesep=1pt]{->}{78}{78}
\nbput[nrot=:U]{a}
\nccurve[angleA=-105,angleB=-55,ncurv=3,nodesep=1pt]{->}{4}{4}
\nbput[nrot=:U]{b}
\endpsmatrix$ }
\end{figure}



\question{5}{ }

\part{a} a (left most)
\begin{itemize}
\item $E \Rightarrow T$
\item $\;\;\;\;\Rightarrow F$
\item $\;\;\;\;\Rightarrow a$
\end{itemize}

\pstree[]{\Tcircle{E} }{
        \pstree{\Tcircle{F}}{ \Tcircle{a} } }


\pagebreak

\part{b} a + a (left most)
\begin{itemize}
\item $E \Rightarrow E + T$
\item $\;\;\;\;\Rightarrow T + T$
\item $\;\;\;\;\Rightarrow F + T$
\item $\;\;\;\;\Rightarrow a + T$
\item $\;\;\;\;\Rightarrow a + F$
\item $\;\;\;\;\Rightarrow a + a$
\end{itemize}

\pstree[]{\Tcircle{E} }{
        \pstree{ \Tcircle{E} }{ \pstree{ \Tcircle{T} }{  \pstree{\Tcircle{F} }{ \Tcircle{a} }}}
        \Tcircle{+}
        \pstree{ \Tcircle{E} }{ \pstree{ \Tcircle{T} }{  \pstree{\Tcircle{F} }{ \Tcircle{a} }}}

}


\pagebreak

\part{c} a + a + a (right most)

\begin{itemize}
\item $E \Rightarrow E + T$
\item $\;\;\;\;\Rightarrow E + F$
\item $\;\;\;\;\Rightarrow E + a$
\item $\;\;\;\;\Rightarrow E + T + a$
\item $\;\;\;\;\Rightarrow E + T + a$
\item $\;\;\;\;\Rightarrow E + F + a$
\item $\;\;\;\;\Rightarrow E + a + a$
\item $\;\;\;\;\Rightarrow T + a + a$
\item $\;\;\;\;\Rightarrow F + a + a$
\item $\;\;\;\;\Rightarrow a + a + a$
\end{itemize}

\pstree[nodesep=1pt,radius=1pt]{\Tcircle{E} }{
        \pstree{ \Tcircle{E} }{ \pstree{ \Tcircle{E} }{
                                   \pstree{ \Tcircle{T} }{ 
                                      \pstree{ \Tcircle{F} }{ \Tcircle{a} }}} 
                                \Tcircle{+} 
                                \pstree{ \Tcircle{T} }{
                                   \pstree{ \Tcircle{F} }{ \Tcircle{a} }} }
        \Tcircle{+}
        \pstree{ \Tcircle{T} }{ \pstree{ \Tcircle{F} }{ \Tcircle{a} }}

}

\pagebreak

\part{d} ((a))

$E \Rightarrow T$

$\;\;\;\;\Rightarrow F$

$\;\;\;\;\Rightarrow (E)$

$\;\;\;\;\Rightarrow (T)$

$\;\;\;\;\Rightarrow (F)$

$\;\;\;\;\Rightarrow ((E))$

$\;\;\;\;\Rightarrow ((T))$

$\;\;\;\;\Rightarrow ((F))$

$\;\;\;\;\Rightarrow ((a))$

\pstree[nodesep=1pt,radius=1pt]{\Tcircle{E} }{
      \pstree{ \Tcircle{T} }{
                     \pstree{ \Tcircle{F} }{ 
                       \Tcircle{$($} 
                       \pstree{ \Tcircle{E} }{ 
                         \pstree{\Tcircle{T} } {
                           \pstree{\Tcircle{F} } {
                            \Tcircle{$($} 
                              \pstree{ \Tcircle{E} }{ 
                                \pstree{\Tcircle{T} } {
                                  \pstree{\Tcircle{F} } {
                                    \Tcircle{a}
                                  }
                                }
                              }
                            \Tcircle{$)$}
                           }
                         }
                       }
                       \Tcircle{$)$}
                     }
                  } 
}

\question{6}{Sipser 2.4(b): $\{ w | w $ starts and ends with the same symbol$\}$ }

$ R \rightarrow 0 R \;|\; 1 R\; |\; \epsilon $


$ S \rightarrow 0\; |\; 1 \;|\; 0 R 0 \;|\; 1 R 1$

\end{document}
