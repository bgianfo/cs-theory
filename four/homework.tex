\documentclass[11pt]{article}

\oddsidemargin0cm
\topmargin-2cm     %I recommend adding these three lines to increase the 
\textwidth16.5cm   %amount of usable space on the page (and save trees)
\textheight23.5cm  

\newcommand{\question}[2] {\vspace{.25in} \fbox{#1} #2 \vspace{.10in}}
\renewcommand{\part}[1] {\vspace{.10in} {\bf (#1)}}

\usepackage{graphicx}
\usepackage{amssymb,amsmath}

\begin{document}

\medskip                        % Skip a "medium" amount of space
                                % (latex determines what medium is)
                                % Also try: \bigskip, \littleskip


\begin{center}                  % Center the following lines
  {\Large Introduction to the Theory of Computation \\ Homework \#4} \\
  Brian Gianforcaro \\
  \date \\
\end{center}

\ttfamily

\question{1}{Sipser: 1.15}
 
\part{a} The states of N are the states of N1. 
\begin{center}
  \begin{itemize}
    \item $N_1 = (Q_1,\Sigma,{\delta}_1,q_1, F_1 = \{ q_1 \}) $
  \end{itemize}
\end{center}


\part{b} The start state of N is the same as the start state of N1. 
\begin{center}
  \begin{itemize}
    \item ${\delta}_1(q1,\epsilon) = q1 $
    \item ${\delta}_1(q1, x \in \Sigma ) = q1 $
    \item $N_1 = (Q_1,\Sigma,{\delta}_1,q_1, F_1) $
  \end{itemize}
\end{center}


\part{c} $F =\{q\}\cup F_1$.  The accept states F are the old accept states plus its start state. 
\begin{center}
  \begin{itemize}
    \item ${\delta}_1(q2,\epsilon) = q2 $
    \item ${\delta}_1(q2, x \in \Sigma ) = q2 $
    \item $N_1 = (Q_1,\Sigma,{\delta}_1, q1, F_1 = \{ q2 \} ) $
  \end{itemize}
\end{center}

\part{d} Define $\delta$ so that for any $q \in Q$ and any $a \in \Sigma$, 
\begin{center}
  \begin{itemize}
    \item $N_1 = (Q_1,\Sigma,{\delta}_1,q_1, F_1) $
  \end{itemize}
\end{center}

\question{2}{Proof: By Contradiction}
\begin{center}
  \begin{itemize}
    \item Suppose $A_2 = $ \{ $ www | w \in \{a,b\}^* $ \}  is regular.
    \item Then the pumping lemma applies, and there exists a number n so that any $x \in A_2$ can be pumped.
    \item Pick $x = a^n b a^n b a^n b $ and $ y = b^m $
    \item Since $|x| \geq n$, it follows that $y = uvw$, $|v| \geq 1$, $x(0) \in A_2$. 
    \item Clearly v is a string of a's.
    \item Hence $x(0) = a^{n-|v|} b a^n b a^n b$
    \item But, since $n \ne  n-|v|$, $x(0) \notin A_2$.
    \item Since $x(0) \in A_2$ and $x(0) \notin A_2$ are contradictory,
    \item L regular can not be the case.
  \end{itemize}
\end{center}

\indent $\Box$

\pagebreak 

\question{3}{Proof: By Contradiction}
\begin{center}
  \begin{itemize}
    \item Suppose $L = $ \{ $a^i b^k | 2k \leq i \leq 3k $ \}  is regular.
    \item Then the pumping lemma applies, and there exists a number n so that any $x \in L$ can be pumped.
    \item Pick $x = a^n $ and $ y = b^m $
    \item Since $|x| \geq n$, it follows that $y = uvw$, $|v| \geq 1$, $x(0) \in L$. 
    \item Clearly v is a string of a's.
    \item Hence $x(0) = a^n b^{m-|v|}$
    \item But, since $m \ne  m-|v|$, $x(0) \notin L$.
    \item Since $x(0) \in L$ and $x(0) \notin L$ are contradictory,
    \item L regular can not be the case.
  \end{itemize}
\end{center}

\indent $\Box$

\question{4}{}

 \part{a} False, {${a,b}^*$} is regular, but it contains a non-regular subset \{$a^n b^n | n \geq 0 $\}

 \part{b} False,  Non-regular languages have finite subsets, and finite languages are regular.

 \part{c} Flase, the union of of a language and it's compliment is ${\Sigma}^{*}$ which is regular.

 \part{d} Flase, A non-regular languages intersected with it's complement is empty, which is regular.

 \part{e} False, ${L}_{2}$ could just be a subset of ${L}_{1}$.

\end{document}
